% !TeX root = ../mainfile.tex

\chapter{Code Boxes}

Reference to \refn[l]{list:regarrays}. The command \code{\\refn} additionally adds a localized typename in front of the number. The localization has to be set inside \code{styling.tex}.

\codebox{verilog}{inc/code/addierzelle_entity_regarray.v}{list:regarrays}{Example of a portlist with register- / wirearrays}

\codeboxdesc[17]{verilog}{inc/code/addierzelle_entity_generic.v}{list:regarrays_gen}{Example of a generic portlist with register- / wirearrays}{This is a codebox with an explicit defined description area. Text only referencing the code can be placed in here. Additionally referencing single lines works like \refnum{line:C_IN} this or like this \refnum{line:referencing}!}

Additionally next to \code{inline highlighting} there are textboxes too:

\highlightbox{0.93}{
	\inlinetext{mex -lde265 decoder/interface.cpp decoder/decoder.cpp -outdir decoder/bin;}
}

\chapter{Images Boxes}

Image boxes can be created in the same style as codeboxes. There are predefined image box styles for single and double images and a generic box for the use with multiple elements.

\imgboxdesc[1]{inc/img/img1.png}{fig:img1}{A single image}{This is a single image that uses the full width of the page. It can have a command in the same style as the codebox uses.}

\doubleimgboxdesc{fig:double}{inc/img/img1.png}{Image 1}{fig:double1}{inc/img/img2.png}{Image 2}{fig:double2}{Two images can be displayed next to each other if they have the same dimensions.}{Two images next to eachother}

\newpage

\chapter{Titles may be hidden!}
In this chapter some titles may be hidden and therefore are not shown in the TOC.

\addtocontents{toc}{\protect\setcounter{tocdepth}{1}}

% From this point on, only show up to \sections in the ToC
\section{Not Hidden Section}
I'm not hidden in the ToC! \cite{Albrecht2010}

\section*{Hidden subsection without a number}
I'm hidden!\cite{Albrecht2010_IEICE}

\addtocontents{toc}{\protect\setcounter{tocdepth}{5}}
% From this point on, show everything again
