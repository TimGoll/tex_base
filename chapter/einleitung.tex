\section{Praktikumsablauf und Einleitung}
\label{sec:einleitung}
In diesem Praktikum wird ein einfacher Dimming-Algorithmus mit Hilfe eines \emph{\textbf{F}ield \textbf{P}rogrammable \textbf{G}ate \textbf{A}rrays} (kurz: \emph{FPGA}) implementiert und auf einem Flüssigkristalldisplay validiert und vermessen. Kenntnisse in \emph{Matlab} sind erforderlich.\\
Die Ziele sind das Kennenlernen der Hardwarebeschreibungssprache \emph{Verilog} und der Funktionsweise von FPGAs.\\
Nach den Einführungskapiteln wird ein komplexeres Digital-Design für ein FPGA implementiert. Dabei wird auf einer virtuellen Pixelpipeline ein Bild in Hardware analysiert und manipuliert um auf einem LCD wiedergegeben zu werden.

\subsection*{Organisatorisches und Ablauf}
Dieses Praktikum findet als zweiwöchiger Blockkurs statt und beginnt täglich pünktlich um 08:30 Uhr im Praktikumsraum (Gebäude A5.1, Raum -1.14 (Untergeschoss)).
Dieses Praktikum ist \mbox{\textbf{benotet}}. Die Noten werden durch obligatorische Minitests festgesetzt. Für das Bestehen müssen diese Tests zwingend bestanden, sowie alle Versuche und Aufgaben erfolgreich abgeschlossen werden. Des Weiteren ist eine Abgabe einer Ausarbeitung zum Ende des Praktikums verpflichtend.\\
Die Aufgabenstellungen zu den jeweiligen Kapiteln werden getrennt von diesem Script ausgegeben.
Die Aufgaben, welche in die Ausarbeitung einfließen müssen, sind entsprechend mit dem Symbol \includegraphics[width=0.03\linewidth]{img/text_ausarbeitung} am Rand gekennzeichnet.\\
Zur Vorbereitung sollen Teile des Scriptes jeweils für die kommenden Termin durchgearbeitet bzw. durchgelesen werden. Der Betreuer gibt hierbei den Scriptteil vor, welcher vorbereitet werden soll. Speichern Sie \emph{alle} Ihre Daten strukturiert und dokumentiert auf Ihrem Desktop in dem Ordner:\\
\centerline{\texttt{\verzeichnisDesktop/*}}\\
Außerdem müssen Sie Ihr Hardwaredesign und Vorgehen in möglichen Code-Reviews erläutern können. Achten Sie auf saubere und verständliche Dokumentation des Codes und fügen Sie in jede Datei Ihren Namen und Matrikelnummer in die Kommentare ein. Plagiieren führt zu einem sofortigen Ausschluss und Nichtbestehen des Praktikums. \\
Auf dem Rechner mit der IP \texttt{134.96.62.241} werden per Samba einige Dateien bereitgestellt. Diese können von dort kopiert werden. Der entsprechende Ordner wird automatisch unter \texttt{/home/student/share} eingebunden und kann vom Desktop aus geöffnet werden.
\newpage
\subsection*{Einleitung}
Der mit Abstand höchste Strombedarf bei einem LC-Displays hat die Hintergrundbeleuchtung (engl. \emph{Backlight}). Der Anteil des Gesamtverbrauchs\footnote{\emph{Gesamtverbrauch} ist Umgangssprache. Energie wird nicht verbraucht.} kann bei 80\,\% liegen. Die Ursachen sind unter anderem die Polfilter, deren Effizienz gering ist, sowie den 3 Farbfilter der Subpixel, welche ebenfalls jeweils $ \approx\frac{2}{3} $ des Lichtes blockieren. Durch diese Umstände muss das Backlight sehr hell sein, um die geforderte Leuchtdichte zu erhalten.
Wird das Bild vor der Darstellung analysiert, kann die Hintergrundbeleuchtung so angepasst werden, dass sie optimal zum Bildinhalt passt. Dadurch wird sie optimalerweise dunkler, wodurch der Strombedarf der Hintergrundbeleuchtung sinkt.
Liegt beispielsweise der hellste Punkt des Bildes bei 50\,\% der maximalen Leuchtdichte, so kann die Transmission der Pixel auf bis zu 100\,\% erhöht und die Hintergrundbeleuchtung reduziert werden, wodurch das angezeigte Bild die selbe finale Leuchtdichte hat wie das Originale Bild. Dies nennt sich \textit{Global-Dimming}.\\
Hat das Display einzeln ansteuerbare LEDs in der Hintergrundbeleuchtung, dann kann mit sogenanntem \textit{Local-Dimming} nicht nur eine Energieeinsparung erreicht, sondern auch der Kontrast verbessert werden. Hierfür wird das Bild analysiert und die Beleuchtung einzelner Subbereiche voneinander getrennt angesteuert, vergleichbar mit der oben beschriebenen Technik.\\
Local-Dimming kann zusätzlich mit einer \emph{globalen} Preprocessor-Stufe in der Leistungseinsparung verbessert werden \cite{Schmidt_SID15}. Interessierte finden in \cite{Chen2007} und \cite{Albrecht2010} weitere Literatur zu Dimming Techniken.