% !TeX root = ../thesis.tex

% https://en.wikibooks.org/wiki/LaTeX/Source_Code_Listings

% MATLAB style
\lstdefinestyle{matlab} {
	escapeinside={/*@}{@*/},
	language=matlab, 
	morekeywords={classdef, properties, methods},
	deletekeywords={gamma, type},
	basicstyle=\ttfamily \small  \color{inlineBoxListingText},
	identifierstyle=\ttfamily \small \color{black},
	keywordstyle=\bfseries \color{keywordcolor},
	commentstyle=\itshape \color{commentcolor},
	stringstyle=\ttfamily \color{stringcolor},
	showstringspaces=false,
	frame=none,
	breaklines=true, 
	numbers=left,
	numberstyle={\llabelput \scriptsize \ttfamily \color{numbercolor}}, 
	numbersep=7.5pt,
	frame=lines,
	rulecolor=\color{bordercolor},
	backgroundcolor=\color{backgroundcolor},
	captionpos=b,
	tabsize=4
}

% VERILOG style
\lstdefinestyle{verilog} {
	escapeinside={/*@}{@*/},
	language=Verilog,
	basicstyle=\ttfamily \small,
	identifierstyle=\ttfamily \small \color{inlineBoxListingText},
	keywordstyle=\bfseries \color{keywordcolor},
	commentstyle=\itshape \color{commentcolor},
	stringstyle=\ttfamily \color{stringcolor},
	showstringspaces=false,
	frame=none,
	breaklines=true, 
	numbers=left,
	numberstyle={\llabelput \scriptsize \ttfamily \color{numbercolor}},
	numbersep=7.5pt,
	frame=lines,
	rulecolor=\color{bordercolor},
	backgroundcolor=\color{backgroundcolor},
	captionpos=b,    
	tabsize=4
}

% VERILOG style
\lstdefinestyle{cpp} {
	escapeinside={/*@}{@*/},
	language=c++,
	basicstyle=\ttfamily \small \color{inlineBoxListingText},
	identifierstyle=\ttfamily \small \color{black},
	keywordstyle=\bfseries \color{keywordcolor},
	commentstyle=\itshape \color{commentcolor},
	stringstyle=\ttfamily \color{stringcolor},
	showstringspaces=false,
	frame=none,
	breaklines=true, 
	numbers=left,
	numberstyle={\llabelput \scriptsize \ttfamily \color{numbercolor}},
	numbersep=7.5pt,
	frame=lines,
	rulecolor=\color{bordercolor},
	backgroundcolor=\color{backgroundcolor},
	captionpos=b,    
	tabsize=4
}

% LISTING NUMBERED LINKS
%define listing counter
\makeatletter
\AtBeginDocument{%
	% Counter `lstlisting' is not defined before `\begin{document}'
	\newcounter{llabel}[lstlisting]%
	\renewcommand*{\thellabel}{%
		\ifnum\value{llabel}<0 %  
		\@ctrerr
		\else
		\ifnum\value{llabel}>10 %
		\@ctrerr
		\else
		\protect\ding{\the\numexpr\value{llabel}+181\relax}%
		\fi
		\fi  
	}%   
}   
\newlength{\llabelsep}

\newcommand*{\llabel@name}{%
	%adding chapter_id*100 to listing_id to remove listings with the same id. Works as long as no chapter as more than 100 listings
	llabel\inteval{\the\value{lstlisting}+\the\value{chapter}*100}.\the\lst@lineno
}
\newcommand*{\llabel}[1]{%
	\begingroup
	\refstepcounter{llabel}%
	\label{#1}%
	\label{\llabel@name}%
	\endgroup
}
\newcommand*{\llabelput}{%
	\@ifundefined{r@\llabel@name}{%
	}{%
		\ifnum\value{lstnumber} < 10
			\setlength{\llabelsep}{6.4pt}
			\raisebox{-1pt}{\small\ref{\llabel@name}}
		\else
			\setlength{\llabelsep}{2pt}
			\raisebox{-1pt}{\small\ref{\llabel@name}}
		\fi
		\kern\llabelsep\scriptsize
	}%
}   
\makeatother

\newcommand{\refnum}[1]{\raisebox{-2pt}{\Large\ref{#1}}}


% INLINE LISTINGS
\def\inlineshorttext{\lstinline[basicstyle=\ttfamily\color{inlineListingText}, keywordstyle=\ttfamily\color{inlineListingText}, identifierstyle=\ttfamily\color{inlineListingText}, stringstyle=\ttfamily\color{inlineListingText}]}
\def\inlinetext{\lstinline[basicstyle=\ttfamily\color{inlineBoxListingText}, keywordstyle=\ttfamily\color{inlineBoxListingText}, identifierstyle=\ttfamily\color{inlineBoxListingText}, stringstyle=\ttfamily\color{inlineBoxListingText}]}
\def\inlinebox{\colorbox{inlineListingBackground}}

% HIGHLIGHT BOXES WITHOUT SYNTAX HIGHLIGHTING
\newcommand{\highlightbox}[2]{
	\begin{center}
		\begin{tcolorbox}[width=#1\textwidth, colback=backgroundcolor, colframe=bordercolor, boxrule=0.5px, arc=2px, outer arc=2.5px, enlarge bottom by=-0.45em]
			#2
		\end{tcolorbox}
	\end{center}
}

% INLINE SYNTAX HIGHLIGHT
\newcommand{\code}[1]{\inlinebox{\inlineshorttext{#1}}}

% CODEBOX WITH SYNTAX HIGHLIGHT
\newcommand{\codebox}[5][1]{
	\begin{figure}[H]
		\lstset{style=#2}
		\lstinputlisting[firstnumber=#1]{#3}
		\captionof{lstlisting}{#5}
		\label{#4}
	\end{figure}
}

\newcommand{\codeboxdesc}[6][1]{
	\begin{figure}[H]
		\lstset{style=#2}
		\lstinputlisting[firstnumber=#1]{#3}
		\raisebox{-2.5pt}{\parbox[t]{\textwidth}{
			\textit{#6\\}
			\textcolor{bordercolor}{\rule{\textwidth}{0.5px}}
		}}
		\captionof{lstlisting}{#5}
		\label{#4}
	\end{figure}
}